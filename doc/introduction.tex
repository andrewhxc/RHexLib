%
% This file is part of RHexLib, 
%
% Copyright (c) 2001 The University of Michigan, its Regents,
% Fellows, Employees and Agents. All rights reserved, and distributed as
% free software under the following license.
% 
%  Redistribution and use in source and binary forms, with or without
% modification, are permitted provided that the following conditions are
% met:
% 
% 1) Redistributions of source code must retain the above copyright
% notice, this list of conditions, the following disclaimer and the
% file called "CREDITS" which accompanies this distribution.
% 
% 2) Redistributions in binary form must reproduce the above copyright
% notice, this list of conditions, the following disclaimer and the file
% called "CREDITS" which accompanies this distribution in the
% documentation and/or other materials provided with the distribution.
% 
% 3) Neither the name of the University of Michigan, Ann Arbor or the
% names of its contributors may be used to endorse or promote products
% derived from this software without specific prior written permission.
% 
% THIS SOFTWARE IS PROVIDED BY THE COPYRIGHT HOLDERS AND CONTRIBUTORS
% "AS IS" AND ANY EXPRESS OR IMPLIED WARRANTIES, INCLUDING, BUT NOT
% LIMITED TO, THE IMPLIED WARRANTIES OF MERCHANTABILITY AND FITNESS FOR
% A PARTICULAR PURPOSE ARE DISCLAIMED. IN NO EVENT SHALL THE REGENTS OR
% CONTRIBUTORS BE LIABLE FOR ANY DIRECT, INDIRECT, INCIDENTAL, SPECIAL,
% EXEMPLARY, OR CONSEQUENTIAL DAMAGES (INCLUDING, BUT NOT LIMITED TO,
% PROCUREMENT OF SUBSTITUTE GOODS OR SERVICES; LOSS OF USE, DATA, OR
% PROFITS; OR BUSINESS INTERRUPTION) HOWEVER CAUSED AND ON ANY THEORY OF
% LIABILITY, WHETHER IN CONTRACT, STRICT LIABILITY, OR TORT (INCLUDING
% NEGLIGENCE OR OTHERWISE) ARISING IN ANY WAY OUT OF THE USE OF THIS
% SOFTWARE, EVEN IF ADVISED OF THE POSSIBILITY OF SUCH DAMAGE.

%%%%%%%%%%%%%%%%%%%%%%%%%%%%%%%%%%%%%%%%%%%%%%%%%%%%%%%%%%%%%%%%%%%%%%
% $Id: introduction.tex,v 1.3 2001/07/18 20:04:50 ulucs Exp $
%
% Created       : Uluc Saranli, 01/06/2001
% Last Modified : Uluc Saranli, 06/27/2001
%
%%%%%%%%%%%%%%%%%%%%%%%%%%%%%%%%%%%%%%%%%%%%%%%%%%%%%%%%%%%%%%%%%%%%%%

\chapter{Introduction}

\section{Overview}

RHexLib is a collection of software components and modules to run on the
RHex robot. It incorporates low level hardware access routines, motion
control functionality and high level controllers. It is designed to be
modular and functional on different versions of underlying hardware with
minimal code modifications.

Currently, RHexLib has four major packages:

\begin{enumerate}
\item{\bf RHexLib\_core}\par 
This package is a collection of core components and the most stable standard
set of controllers. It also incorporates all the low level hardware
libraries for different versions of the robot. The following list summarizes
the contents of this package.

\begin{itemize}

\item{The Module Manager : A static scheduler for
periodic tasks}

\item{Symbol Table facilities: Management of configuration files and access
to configuration symbols.}

\item{Data logging tools : Modules to manage runtime logging and
storage of data.}

\item{State Machine framework : Formal mechanisms for defining and executing
state machines.}

\item{Basic motion control tools : Motor position sensing, local PD and
torque control, generic motion profiling tools.}

\item{Basic RHex calibration : Modules to perform automatic actuator and
sensor calibration of the robot.}

\item{Basic RHex controllers : Standing, sitting, alternating tripod walking
and alternating triwheel gait controllers.}

\item{Framework for hardware access : An abstract  class interface to low level hardware.}

\item{Hardware libraries : Support for Michigan RHex, McGill RHex, interface
to SimSect++ simulation environment and a set of virtual free running
motors.}

\item{Various examples : Lots of examples, module and template controllers.}
\end{itemize}

The components in this package are unlikely to undergo any significant
changes. New releases of the library will possibly add new components or
improve the existing ones in a backwards compatible way.

\item{\bf SimSect++}\par

SimSect++ is a simulation package for RHex, developed by Uluc Saranli. This
package is a version of SimSect++, which works with RHexLib in a transparent
way to implement the ability of testing RHexLib code outside the robot
itself. Being a somewhat simplified model of the robot, the behavioral
results of thesimulation are not as accurate, but as a development
framework, it is invaluable.

\item{\bf RHexLib\_Michigan}\par
This package contains all the modules and examples as well as the common
demo executable for the Michigan development team. The components in this
library are not essential to RHexLib's existence.

The idea behind this institution specific package is to decouple the
experimental development efforts between different institutions and minimize
the impact of such development on others. As such, this local package is
less constraining than the core library, but still has some restrictions on
minimal functionality before admitting a component.

\item{\bf RHexLib\_McGill}\par
This package contains all the modules and examples as well as the common
demo executable for the McGill development team. Its structure is the same
as the Michigan local package.

The latest versions of each of these packages as well as this documentation
are available through {\tt rhex.sourceforge.net}.
\end{enumerate}

\section{About This Manual}

We start in Chapter \ref{sec:quickstart} with tutorials on many commonly
encountered tasks, with specific worked out examples. Chapter
\ref{sec:fundamentals} then details the fundamentals of the library,
including all the library specific data types. This is followed by a
detailed description of the module manager and related tools in Chapter
\ref{sec:modulemanager}, which constitutes the foundations of RHexLib.

Chapter \ref{sec:rhexlib_module_summary} then gives a detailed summary of
all the modules in the core library, followed by Chapter
\ref{sec:utilities}, a reference for the various utilities in the
library. Detailed descriptions of the standard hardware components supported
by the abstract Hardware base class are presented in Chapter
\ref{sec:hardware_interface}. Various implementations of this hardware
interface in different hardware libraries are then described in Chapter
\ref{sec:hardware_libraries}. 

