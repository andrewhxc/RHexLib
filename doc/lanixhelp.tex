%
% This file is part of RHexLib, 
%
% Copyright (c) 2001 The University of Michigan, its Regents,
% Fellows, Employees and Agents. All rights reserved, and distributed as
% free software under the following license.
% 
%  Redistribution and use in source and binary forms, with or without
% modification, are permitted provided that the following conditions are
% met:
% 
% 1) Redistributions of source code must retain the above copyright
% notice, this list of conditions, the following disclaimer and the
% file called "CREDITS" which accompanies this distribution.
% 
% 2) Redistributions in binary form must reproduce the above copyright
% notice, this list of conditions, the following disclaimer and the file
% called "CREDITS" which accompanies this distribution in the
% documentation and/or other materials provided with the distribution.
% 
% 3) Neither the name of the University of Michigan, Ann Arbor or the
% names of its contributors may be used to endorse or promote products
% derived from this software without specific prior written permission.
% 
% THIS SOFTWARE IS PROVIDED BY THE COPYRIGHT HOLDERS AND CONTRIBUTORS
% "AS IS" AND ANY EXPRESS OR IMPLIED WARRANTIES, INCLUDING, BUT NOT
% LIMITED TO, THE IMPLIED WARRANTIES OF MERCHANTABILITY AND FITNESS FOR
% A PARTICULAR PURPOSE ARE DISCLAIMED. IN NO EVENT SHALL THE REGENTS OR
% CONTRIBUTORS BE LIABLE FOR ANY DIRECT, INDIRECT, INCIDENTAL, SPECIAL,
% EXEMPLARY, OR CONSEQUENTIAL DAMAGES (INCLUDING, BUT NOT LIMITED TO,
% PROCUREMENT OF SUBSTITUTE GOODS OR SERVICES; LOSS OF USE, DATA, OR
% PROFITS; OR BUSINESS INTERRUPTION) HOWEVER CAUSED AND ON ANY THEORY OF
% LIABILITY, WHETHER IN CONTRACT, STRICT LIABILITY, OR TORT (INCLUDING
% NEGLIGENCE OR OTHERWISE) ARISING IN ANY WAY OUT OF THE USE OF THIS
% SOFTWARE, EVEN IF ADVISED OF THE POSSIBILITY OF SUCH DAMAGE.

%%%%%%%%%%%%%%%%%%%%%%%%%%%%%%%%%%%%%%%%%%%%%%%%%%%%%%%%%%%%%%%%%%%%%%
% $Id: lanixhelp.tex,v 1.2 2001/07/12 17:14:10 ulucs Exp $
%
% *nix tutorial.
%
% Created       : Laura McWilliams, 05/12/2001
% Last Modified : Laura McWilliams, 05/27/2001
%
%%%%%%%%%%%%%%%%%%%%%%%%%%%%%%%%%%%%%%%%%%%%%%%%%%%%%%%%%%%%%%%%%%%%%%

\documentclass[12pt, letterpaper]{article}

\usepackage{hhline}
\usepackage{alltt}

\setlength{\paperwidth}{8.5in}
\setlength{\paperheight}{11in}


\begin{document}


\title{\bf {\Large Laura's Abbreviated Guide to *NIX commands}\\ 
           {\large (for new Kodlab members)}}
\author{\large \it Laura McWilliams (lmcwil@umich.edu)}
\maketitle 

\section*{Introduction}
Some people who come to KodLab have never used Linux before,
so I figured it might be helpful to have a quick brief reference
available to help you figure out how to do basic things.

\section{Use tcsh}
tcsh is cool, in my opinion. It's nice because you can do spiffy
things. For instance,
hitting TAB tries to figure out what you were starting to type (so you 
don't have to always type out the whole thing.
You can use the arrow keys to scroll through all the commands you've
typed in at this shell.
So that's why they pretty much set up everybody initially with ctsh as 
default shell to use. If not, you can type tcsh and you'll be using
tcsh. I wish I knew how all this shell stuff worked; then I could
explain it all to you.

\section{Basic Commands}

\subsection{Moving Around}
To figure out {\it where} you are, type {\bf pwd} (for Print Working
Directory).\\
To move to a different directory, type {\bf cd pathname} where
pathname is the path to the new directory, relative to where you are
now, or an absolute pathname. For instance, if the directory you want
to go to is\\
 {\bf /yildiz/u/uniqname/RHex/blah/ }\\
 but you're currently at {\bf /yildiz/u/uniqname/somewhere/ },\\
 you could type {\bf cd /yildiz/u/uniqname/RHex/blah }\\
 or simply use the relative path
and type {\bf cd ../RHex/blah }.\\
In Unix, a single dot {\bf .} represents the directory you are in,
while two dots {\bf ..} means the directory one level above. 

\subsection{Messing with Files}
To copy stuff, use {\bf cp filename newfilename }. NOTE: with this, as 
with other commands, you can specify a path if you're not right in the 
directory with the files you want to manipulate. For instance, if
you want to copy file goo.txt to the directory two levels above, you
can type {\bf cp goo.txt ../../ } and it will automatically give it
the same name, goo.txt in the new location.\\
To move a file, use {\bf mv oldfilename newfilename }, which is the
same as copy, only it gets rid of the original version.\\
To remove a file, type {\bf rm filename }. Don't do this unless you
mean it.\\

\subsection{Running Programs}
To run a program, simply type the name of the program at the
prompt. If you are running an executable you created yourself, you
should type {\bf ./myexecutable } to run the program.\\
If you want to run a process in the background (that is, the program
will run, but your shell will still be able to do other stuff at the
same time), follow the command with an ampersand {\bf \&}. For
instance, you could type {\bf xemacs foo.cc \& }. That way, you can
still use the same shell to run other processes while you have xemacs
open to edit foo.cc.


\section{Wildcards and Recursion}

\subsection{Wildcards}
In Unix, you can use star {\bf *} as a wildcard. So you could say {\bf 
cp *.cc ../backup/ } to copy all your .cc files to another
directory. Or, you could say {\bf rm *.o } to get rid of all the
object files in the current directory. It's a neat feature.

\subsection{Recursion}
If you want to do a regular UNIX file command on a directory, you must 
call it recursively. This can be done with basic commands by typing
{\bf -R} or {\bf -r} after the command. For example, saying {\bf cp -r 
mydirectory ../backup} will copy the {\it entire directory}, not only
its contents, into the directory {\bf backup}. So the copy will be
{\bf backup/mydirectory}.


\section{Secure Shell}
For security purposes, the KodLab computers are set up so you can't
use telnet or ftp to access them.
To use secure shell, type {\bf ssh hostname }. It will ask you for
your password. You can use this to get onto local computers, like {\it 
poyraz} or {\it serefe} (my personal favorite) or external hosts, like 
{\it login.itd.umich.edu}.

\subsection{Secure Copy}
The syntax of Secure Copy is obscure and often screwed up by people
like me. If you are simply sitting on your own computer, you can type:
{\bf scp uniqname@originhost:filename
uniqname@destinationhost:filename }. It will ask you for passwords on
both hosts.\\
It seems to me that often you will already be ssh'd to the computer
where you are copying files to and/or from, in which case, you don't
have to type in all that info. For example, let's say you are ssh'd to 
your afs space which is on login.itd.umich.edu, or afs.engin.umich.edu 
or whatever it is you engineering students save all your stuff
on. Suppose you want to copy myfile.txt over from your afs into your
home directory. You could
just type {\bf scp myfile.txt uniqname@thiscomputer: } and it will put 
myfile.txt into your home directory on the computer named {\bf
thiscomputer}. It will ask you for your password.

\end{document}
