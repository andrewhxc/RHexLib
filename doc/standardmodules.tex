%
% This file is part of RHexLib, 
%
% Copyright (c) 2001 The University of Michigan, its Regents,
% Fellows, Employees and Agents. All rights reserved, and distributed as
% free software under the following license.
% 
%  Redistribution and use in source and binary forms, with or without
% modification, are permitted provided that the following conditions are
% met:
% 
% 1) Redistributions of source code must retain the above copyright
% notice, this list of conditions, the following disclaimer and the
% file called "CREDITS" which accompanies this distribution.
% 
% 2) Redistributions in binary form must reproduce the above copyright
% notice, this list of conditions, the following disclaimer and the file
% called "CREDITS" which accompanies this distribution in the
% documentation and/or other materials provided with the distribution.
% 
% 3) Neither the name of the University of Michigan, Ann Arbor or the
% names of its contributors may be used to endorse or promote products
% derived from this software without specific prior written permission.
% 
% THIS SOFTWARE IS PROVIDED BY THE COPYRIGHT HOLDERS AND CONTRIBUTORS
% "AS IS" AND ANY EXPRESS OR IMPLIED WARRANTIES, INCLUDING, BUT NOT
% LIMITED TO, THE IMPLIED WARRANTIES OF MERCHANTABILITY AND FITNESS FOR
% A PARTICULAR PURPOSE ARE DISCLAIMED. IN NO EVENT SHALL THE REGENTS OR
% CONTRIBUTORS BE LIABLE FOR ANY DIRECT, INDIRECT, INCIDENTAL, SPECIAL,
% EXEMPLARY, OR CONSEQUENTIAL DAMAGES (INCLUDING, BUT NOT LIMITED TO,
% PROCUREMENT OF SUBSTITUTE GOODS OR SERVICES; LOSS OF USE, DATA, OR
% PROFITS; OR BUSINESS INTERRUPTION) HOWEVER CAUSED AND ON ANY THEORY OF
% LIABILITY, WHETHER IN CONTRACT, STRICT LIABILITY, OR TORT (INCLUDING
% NEGLIGENCE OR OTHERWISE) ARISING IN ANY WAY OUT OF THE USE OF THIS
% SOFTWARE, EVEN IF ADVISED OF THE POSSIBILITY OF SUCH DAMAGE.

%%%%%%%%%%%%%%%%%%%%%%%%%%%%%%%%%%%%%%%%%%%%%%%%%%%%%%%%%%%%%%%%%%%%%%
% $Id: standardmodules.tex,v 1.9 2001/07/26 05:18:40 ulucs Exp $
%
% Created       : Uluc Saranli, 01/06/2001
% Last Modified : Uluc Saranli, 06/27/2001
%
%%%%%%%%%%%%%%%%%%%%%%%%%%%%%%%%%%%%%%%%%%%%%%%%%%%%%%%%%%%%%%%%%%%%%%

\chapter{Summary of Standard RHexLib Modules}
\label{sec:rhexlib_module_summary}

This section outlines the interfaces and properties of all standard modules
provided with \rhexlib. At the end, Figure \ref{fig:module_summary} presents 
a general view of how these components fit together.

\section{Standard Module Utilities}

In addition to providing a standard set of modules, RHexLib also provides
some utilities to facilitate their use.

\begin{codesegment}
#include "StdModules.hh"
\end{codesegment}

\begin{prototype}
void RHexAddStdModules( void );
\end{prototype}

This function adds all the standard RHexLib modules to the module manager's
list with appropriate periods, offsets and orders. The names, and
scheduling parameters of all the standard modules are defined in {\tt
  StdModules.hh}

\begin{prototype}
void RHexRemoveStdModules( void );
\end{prototype}

This function removes all the standard RHexLib modules to the module manager's
list. Note that this function can only be called after {\tt
  RHexAddStdModules()} is called first.


The following macros define the base orders for different types of
modules. For proper functioning of the system, these macros should be used
when adding modules to be used with the standard set of RHexLib modules.

\begin{datatype}
SENSING\_MODULES \\
USER\_DEFINED\_SENSORS \\
USER\_CONTROLLERS \\
BEHAVIORAL\_CONTROLLERS \\
LOW\_LEVEL\_CONTROLLERS \\
ACTUATOR\_MODULES \\
LOGGING\_MODULES \\
OTHER\_MODULES \\
\end{datatype}

\section{Basic Modules}

\subsection{The \EncoderReader\ Module}

\begin{moduleheader}
\classname{\EncoderReader} \mline
\modulebase{\Module} \mline
\modulename{encreader} \mline
\moduleflags{multi-user, non-polling}
\usedmodules{None}
\end{moduleheader}

The \EncoderReader\ module periodically reads the current count of the
encoder corresponding to its {\tt index} from the system level \Hardware\
object and keeps track of an absolute angular position and velocity for the
corresponding axis. \\

\constructors

\begin{itemize}
\item{\tt EncoderReader ( int index ); }
\end{itemize}

\localinterface

\begin{itemize}
\item{\tt void reset( float pos );}\par
This method sets the current position to {\tt pos}, in radians.
\item{\tt float getPosition( void );} \par 
This method returns the current position in radians, in the range $[-\pi,
\pi]$.
\item{\tt float getSpeed( void );} \par
This method returns the current speed in $rad/s$.
\item{\tt \CLOCK\ getTimestamp( void );} \par
This method returns the timestamp of the latest encoder read.
\end{itemize}

\subsection{The \SpeedFilter\ Module}

\begin{moduleheader}
\classname{\SpeedFilter} \mline
\modulebase{\Module} \mline
\modulename{spdfilter} \mline
\moduleflags{multi-user, non-polling}
\usedmodules{\EncoderReader}
\end{moduleheader}

The \SpeedFilter\ module periodically reads the speed from the
\EncoderReader\ module with the same {\tt index}, and filters the value with
a simple FIR filter. \\

\constructors

\begin{itemize}
\item{\tt SpeedFilter ( int index ); }
\end{itemize}

\localinterface

\begin{itemize}
\item{\tt float getSpeed( void );}\par
This method returns the current filtered speed in $rad/s$.
\end{itemize}

\configsymbols\ None.

\subsection{The \AnalogOutput\ Module}

\begin{moduleheader}
\classname{\AnalogOutput} \mline
\modulebase{\Module} \mline
\modulename{analogout} \mline
\moduleflags{single-user, non-polling}
\usedmodules{None.}
\end{moduleheader}

The \AnalogOutput\ module periodically outputs its current value on the
analog output channel corresponding to its { \tt index} on the system level
\Hardware\ object. \\

\constructors

\begin{itemize}
\item{\tt AnalogOutput ( int index ); }
\end{itemize}

\localinterface

\begin{itemize}
\item{\tt void setValue( float v );}\par
This method sets the current output to {\tt v}, in Volts.
\item{\tt float getValue( void );}\par
This method returns the current output value in Volts.
\end{itemize}

\configsymbols\ None.


\section{Motor Control Modules}

\subsection{The \MotorControl\ Abstract Base Class}

\begin{moduleheader}
\classname{\MotorControl} \mline
\modulebase{\Module} \mline
\moduleflags{single-user, non-polling}
\end{moduleheader}

The \MotorControl\ class is a abstract base class for all modules that
achieve feedback motor control. It enforces a uniform interface to all such
modules. Inherited classes must still provide the {\tt init()}, {\tt
uninit()}, {\tt activate()}, {\tt deactivate()}, {\tt update()} methods.  \\

\constructors

\begin{itemize}
\item{\tt MotorControl( char *name, int index ); }
\end{itemize}

\localinterface

\begin{itemize}
\item{\tt void getGains( MotorGains\_t *g );}\par
This method returns the current motor gains in the struct *g.
\item{\tt void setGains( MotorGains\_t *g );}\par
This method sets the current motor gains to the contents of *g.
\item{\tt void getTarget( MotorTarget\_t *t );}\par
This method returns the current motor target in the struct *t.
\item{\tt void setTarget( MotorTarget\_t *t );}\par
This method sets the current motor target to the contents of *t.
\end{itemize}


\subsection{The \PositionControl\ Module}

\begin{moduleheader}
\classname{\PositionControl} \mline
\modulebase{\Module} \mline
\modulename{poscontrol} \mline
\moduleflags{single-user, non-polling}
\usedmodules{\EncoderReader, \AnalogOutput}
\end{moduleheader}

The \PositionControl\ module performs standard proportional derivative
control over the DC motor corresponding to the module's index number. In
addition to performing position and velocity feedback, it also supports
feedthrough torque commands as well as the ability to configure the type
feedforward compensation.\\

The computed torque command takes the following form.

\begin{equation}
\tau = K_p * e_p + K_d * e_d + \tau_0
\label{eq:torque_feedback}
\end{equation}

where $K_p$ and $K_d$ are the P and D gains, $e_p$ and $e_d$ are the
position and velocity errors and $\tau_0$ is the feedthrough torque setting.

The computation of the position error, due to the rotational nature of the
motor shaft coordinates, requires the definition of an {\em error offset},
$\phi_0$, which is used to determine the range to constrain the position
error term in the above equation.

As a consequence, the cotnroller determines $e_p$ by properly choosing the
integer $k$ in the following equation.

\begin{equation}
e_p = ( \phi_{goal} - \phi ) + 2 k \pi 
\hspace{0.1in}\text{ such that } \hspace{0.1in} 
e_p \in [-\pi+\phi_0, \pi+\phi_0]
\label{eq:pos_error_offset}
\end{equation}

This leverage over the range of the position error on $S_1$ enables features
such as favoring positive ($\phi_0 > 0$) or negative rotation ($\phi_0 < 0$)
of the motor for large position errors (avoiding springback of the leg). Note
that $\phi_0$ cannot be too close to either $\pi$ or $-\pi$ because then,
even when the position of the leg is slightly different than the desired
position, the leg might try to rotate from an unreasonable
direction. Usually, choosing $\phi_0 \in [-\pi/2, \pi/2]$ is a good idea.

Once the torque command is computed, the desired voltage command at motor
terminals is computed using the torque constant of the motor as well as a
feedforward compensation term depending on the current compensation
mode. Currently supported modes are {\tt NONE} and {\tt BACKEMF}.\\

\constructors

\begin{itemize}
\item{\tt PositionControl ( int index ); }
\end{itemize}

\localinterface

\begin{itemize}
\item{\tt void setTorqueOffset( float torque );}\par
This method sets the feedthrough torque value $\tau_0$ of Equation
\ref{eq:torque_feedback}.

\item{\tt void setErrorOffset( float offset );}\par
This method sets the position error offset value $\phi_0$ of Equation
\ref{eq:pos_error_offset}.

\item{\tt float getErrorOffset( void );}\par
This method returns the current setting of $\phi_0$.

\item{\tt void setCompensationType( CompensationType type );}\par
This method sets the type of the feedforward voltage compensation in
computing the motor terminal voltage to be commanded to the physical
hardware. Currently, the position control module assumes that the low level
motor drive circuitry operrates in voltage mode. This requires feedforward
compensation to obtain approximate torque control. The currently supported
types, {\tt NONE} and {\tt BACKEMF} can be used to select the type of the
compensation used by the controller.

\item{\tt CompensationType getCompensationType( void );}\par
This method returns the current type of feedforward voltage compensation.

\item{\tt void freeze( void );}\par
This method freezes the motor position by setting the current motor target 
to the current motor position.
\end{itemize}

\configsymbols\ None.

\section{State Machine Tools}

\subsection{The \StateMachine\ Abstract Base Class}

The \StateMachine\ class is the abstract base class for all state machine
implementations. It defines a uniform interface for defining and running
state machines, automatically handling event checking and state
transitions. There are three related classes: {\tt State}, {\tt Event} and
{\tt Arc}. The first two are abstract base classes. A {\tt StateMachine}
class contains a list of its {\tt State} classes, which are all derived from
the state class. Each state contains a list of arcs, which consist of an
{\tt Event} object and another {\tt State}.

The {\tt Event} abstract base class has the following form.

\begin{moduleheader}
\classname{{\tt Event} } \mline
\modulebase{None} \mline
\end{moduleheader}

\constructors

\begin{itemize}
\item {\tt Event( StateMachine * p );}\par
{\tt p} should point to the {\tt StateMachine} class that owns the event
\end{itemize}

\localinterface

\begin{itemize}
\item{\tt virtual bool check( void );} \par
must be defined by children classes
\end{itemize}

\noindent Programmers can use the {\tt EventObject} macro in the following
way to define new events classes derived from {\tt Event}. This is usually
done in class definition of a machine derived from a state machine so that
the scope of the new class is limited to the derived state machine class.

\begin{codesegment}
  EventObject( NewEvent ) * newEvent;
\end{codesegment}

\noindent This defines a new class called NewEvent which is derived from
{\tt Event}. It declares itself a friend of the containing class, and then
defines {\tt newEvent} to be a pointer to a {\tt NewEvent} object. It is a
programming conventon in RHexLib that event classes are capitalized and
event objects have the same name only with the first symbol in lowercase.

Arcs or {\em transitions} are instantiations of the {\tt Arc} class:

\begin{moduleheader}
\classname{{\tt Arc}} \mline
\modulebase{None} \mline
\end{moduleheader}

\constructors

\begin{itemize}
\item {\tt Arc( State * s, Event * e );}
\end{itemize}

\localinterface

\begin{itemize}
\item{\tt State * getTarget( void );} \par
returns the destination state of the arc 
\end{itemize}

Finally, states must be derived from the {\tt State} abstract base class. 

\begin{moduleheader}
\classname{{\tt State}} \mline
\modulebase{None} \mline
\end{moduleheader}

\constructors

\begin{itemize}
\item{\tt State( StateMachine * p );}
\end{itemize}

\localinterface

\begin{itemize}
\item{\tt void addArc ( Arc * a );} \par
This method adds the arc {\tt a} to the state's outgoing arcs.
\item{\tt Arc * getActiveArc ( void );} \par
Returns first arc leaving state whose event is true (or {\tt NULL} if no
such arc)
\item{\tt virtual void entry ( void );} \par
must be defined by child classes
\item{\tt virtual void during ( void );} \par
must be defined by child classes
\item{\tt virtual void exit ( void );} \par
must be defined by child classes
\end{itemize}

\noindent Programmers may use the {\tt StateObject} macro to define new
state classes within the scope of a {\tt StateMachine} derived class as
follows:

\begin{codesegment}
  StateObject( NewState ) * newState;
\end{codesegment}

\noindent The naming convention for derived state classes is the same as for
events. Programmers may add new outgoing arcs to a state using the {\tt
addArc} method or more conveniently using the {\tt Transition} macro as
follows:

\begin{codesegment}
  Transition ( fromState, anEvent, toState );
\end{codesegment}

\noindent which is syntactic sugar for

\begin{codesegment}
  fromState->addArc ( toState, anEvent );
\end{codesegment}

The {\tt StateMachine} class is as follows.

\begin{moduleheader}
\classname{\StateMachine} \mline
\modulebase{\Module} \mline
\moduleflags{single-user, non-polling}
\usedmodules{None}
\end{moduleheader}

The \AnalogOutput\ module periodically outputs its current value on the
analog output channel corresponding to its { \tt index} on the system level
\Hardware\ object. \\

\constructors

\begin{itemize}
\item{\tt StateMachine( char *n );}
\item{\tt StateMachine( char *n, State *initial );}
\item{\tt StateMachine( char *n, uint index );}
\item{\tt StateMachine( char *n, uint index, State *initial );} 
\end{itemize}

\localinterface

\begin{itemize}
\item{\tt void initialize( State *i );} \par
This method sets the initial state to *i.
\end{itemize}

\noindent Note that state machines are modules and thus have the methods
{\tt activate}, {\tt update} and so on, except that they are not virtual and
thus derived classes do not have to define them, although in practice they
usually do. The {\tt update} function of a {\tt StateMachine} class is quite
simple and is shown below. The private variable {\tt current} points to the
currently active state.

\begin{codesegment}
  void StateMachine::update( void ) {

    Arc * a = current->getActiveArc();

    if ( a != NULL ) {

      current->exit();
      current = a->target;
      current->entry();
    }

    current->during();
  }
\end{codesegment}

\noindent Thus, a {\tt StateMachine} acts like its currently active state
{\em and} checks for active arcs leaving the current state, switching states
to the destination state of the arc if it finds any.

\section{Controller Suite}

\subsection{The \CalibMachine\ Module}
\label{sec:calib_machine}

\begin{moduleheader}
\classname{\CalibMachine} \mline
\modulebase{\StateMachine} \mline
\modulename{calibmachine} \mline
\moduleflags{single-user, non-polling}
\usedmodules{\AnalogOutput, \EncoderReader, \SpeedFilter, \StallSensor,
  \PositionControl.}
\end{moduleheader}

The \CalibMachine\ module is a controller that performs motor drive and
encoder homing calibration. Upon activation, it detects motor polarity,
calibrates the encoders to the correct absolute position and finally
determines the analog output voltage for zero command.\\

Figure \ref{fig:calibmachine} illustrates the state transition diagram for
the calibration state machine. The shaded regions indicate states related to
polarity, encoder and drive calibration algorithms.

%%%%%%%%%%%%%%%%%%%%%%%%%%%%%%%%%%%%%%%%%%%%%%%%
\begin{figure}[!ht]
  \centering
  \epsfig{figure=CalibMachine.eps,width=5.8in}
  \caption{\CalibMachine\ state transition diagram.}
  \label{fig:calibmachine}
\end{figure}
%%%%%%%%%%%%%%%%%%%%%%%%%%%%%%%%%%%%%%%%%%%%%%%%

The polarity calibration is designed to determine the connection polarity of
the motor terminals. Incorrect polarity information will result in unstable
behavior in the local PD feedback control. The algorithm to determine the
polarity is very simple. It starts by applying a positive voltage to the
motor terminals, recording rotational velocity, determining connection
polarity. If, however, the leg is stuck on the ground and is unable to
rotate (stallEv), then it retries with a negative voltage, which will yield
the polarity information if the leg is not permanently stuck.

The calibration of the encoders is a good deal more complicated. The manual
calibration is the simplest. The {\tt encManualCalib} state simply assumes
that the current position of the legs corresponds to leg angles specified in
the {\tt manual\_offset} configuration variable and resets the encoders
accordingly.

In contrast, ground calibration rotates the legs in the positive direction
until they hit the ground, and then resets the encoder angles assuming that
the legs are at angles specified by the {\tt ground\_offset} configuration
variable.

Finally, the home switch calibration relies on the presence of home switches
triggered at a certain angle of the leg. The calibration machine assumes
that the {\tt SwitchHW} component of the low level hardware provides the
outputs of these switches at indices 8 through 13, so that their value is
{\tt true} when the leg triggers the switch. The algorithm starts by
rotating the legs in the positive direction until they either hit the ground
or trigger the switch. If they trigger the switch, it waits until the leg
clears the switch, then starts rotating in the opposite direction to
approach the switch from the correct side. This is important because the leg
angle triggering the switch from different rotational directions will be
different. If, on the other hand, the leg hits the ground, the algorithm
immediately starts seeking the switch by rotating the leg in the negative
direction. 

When the switch is detected while the leg is rotating in the negative
direction, the encoder angles are reset with the angles specified by the
configuration variable {\tt switch\_offset}.

\constructors

\begin{itemize}
\item{\tt CalibMachine( int index ); }
\end{itemize}

\localinterface

\begin{itemize}
\item{\tt CalibStatus getStatus( void ); } \par
This method returns the current calibration status.

\item{\tt void reset( void );} \par
This method invalidates the polaritym encoder and drive calibrations,
causing the machine to perform calibration after the next activation. Note
that the invalidation of the polarity depends on the compile time
configuration of CalibMachine which by default disables the polarity
detection unless the low level hardware does not provide polarity information.

\item{\tt void reset( bool polarity, bool encoder, bool drive );} \par
This method selectively invalidates polarity, encoder and/or drive
calibration. Note that the invalidation of the polarity depends on the
compile time configuration of CalibMachine which by default disables the
polarity detection unless the low level hardware does not provide polarity
information.

\item{\tt void setMode( CalibMode m ); } \par
This method chooses the type of calibration to be used in the next
activation.

\item{\tt CalibMode getMode( void ); } \par
This method returns the current calibration mode

\item{\tt void setUpside( bool down ); } \par
This method chooses whether the body should be assumed to be upside down or
not.  If already calibrated, it resets the encoder reader modules to reflect
the change.

\item{\tt bool isUpsideDown( void ); } \par
This method returns the current assumption for the body orientation.

\end{itemize}

\datatypes

\begin{itemize}
\item{\tt typedef enum CalibMachine::CalibStatus \{ \par
CalibMachine::CALIBRATING, \par
CalibMachine::FAILURE, \par
CalibMachine::SUCCESS  \par \}}\par

This datatype is used to indicate the current status of the calibration
state machine.

\item{\tt typedef enum CalibMachine::CalibMode \{ \par
CalibMachine::MANUAL, \par
CalibMachine::GROUND, \par
CalibMachine::SWITCH \par \} }\par

This datatype is used to configure the calibration state machine to use one
of the supported calibration modes. Currently, three modes are supported:
{\em manual calibration}, which assumes that the legs are statically
positioned by the operator to a predetermined angle; {\em ground
  calibration}, which brings the legs forward to touch the ground and
assumes a particular offset for the legs when they are n contact wioth the
ground, and finally, {\em home switch calibration}, which uses Hall effect
sensors on the hips to calibrate the encoder offsets.
\end{itemize}

\configsymbols

\begin{itemize}

\symbolitem{\tt manual\_offset}\par
This numeric array symbol configures the angles of all six legs when the
legs are manually positioned for manual encoder calibration.

\symbolitem{\tt ground\_offset}\par
This numeric array symbol configures the angles of all six legs when the
legs are touching the ground towards the back of the robot when the robot's
right side is up. These angles are used in the ground encoder calibration.

\symbolitem{\tt switch\_offset}\par
This numeric array symbol configures the angles of all six legs when the
legs trigger the home switch (hall effect sensor) while rotating in the
negative direction. These angles are used during home switch encoder
calibration.

\symbolitem{\tt calib\_timeout}\par
Maximum allowed time for calibration attempts (usec)

\symbolitem{\tt calib\_command} \par
Voltage command for the ground seeking behavior and drive calibration (V)

\symbolitem{\tt calib\_min\_speed} \par
Minimum required motor speed for polarity calibration (rad/s)

\symbolitem{\tt calib\_pol\_time} \par
Minimum required time of nonzero rotation for polarity calibration (usec)

\symbolitem{\tt calib\_stall\_time} \par
Stall time threshold for ground encoder calibration (usec)

\symbolitem{\tt calib\_stall\_toler} \par
Maximum allowed motor rotation speed, still interpreted as stall (rad/s)

\symbolitem{\tt calib\_grnd\_retry} \par
Flag to indicate whether the home switch encoder calibration should fallback
to ground encoder calibration upon failure.

\end{itemize}

\subsection{The \WalkMachine\ Module}
\label{sec:walk_machine}

\begin{moduleheader}
\classname{\WalkMachine} \mline
\modulebase{\StateMachine} \mline
\modulename{walkmachine} \mline
\moduleflags{single-user, non-polling}
\usedmodules{\PositionControl, \EncoderReader.}
\end{moduleheader}

The \WalkMachine\ module is a controller which implements the two-stroke
alternating tripod controller, parametrized by the stride period as well as
duty factors, sweep angles and leg offsets for individual legs. It supports
walking forward and backward, turning in place and differential turning,
even when the robot is upside down. All the functionality can be configured
through the module's local interface. As such, the \WalkMachine\ does not
make any adjustments such as responding to the remote control etc. itself,
and needs a ``supervisor'' to modulate its behavior. \\

\constructors

\begin{itemize}
\item{\tt WalkMachine( void ); }
\end{itemize}

\datatypes

The WalkMachine defines a class datatype to encapsulate the standard
parameters of walking behavior. It takes the form of a class with proper
copy constructors to correctly handle copying of its instances. All of its
member variables are defined public and can be accessed by anyone.

\begin{classdef}
class WalkParam_t {

public:
  WalkParam_t( void ) { };
  WalkParam_t( WalkParam_t & );

  WalkParam_t& operator=( const WalkParam_t & );

public:

  float dutyFactor[6];   // Ratio of slow time to period of rotation
  float sweepAngle[6];   // Angle span during the slow leg sweep
  float legOffset[6];    // Angle offset for the whole profile
  MotorGains_t gains[6]; // Walking motor gains

  double cpgPeriod;      // Period of the whole step cycle    
  double tripodTime;     // Time to prepare and bring down the first tripod
  double standAdjTime;   // Leg offset adjustments time interval while standing
  float turnOffset;      // Differential leg offset adjustment for turning 
  float turnSweepAngle;  // Differential sweep angle adjustment for turning
  float turnDutyFactor;  // Differential duty factor adjustment for turning
  float smooth;          // Ratio for the smoothing speed transitions
  bool cubic;            // Flag to choose between linear or cubic spline smoothing
};
\end{classdef}

\localinterface

\begin{itemize}
\item{\tt void setSpeedCommand( float cmd ); } \par
This method sets the speed command of the walking controller. This
command must be either $-1$, $0$, or $1$.
\item{\tt void setTurnCommand( float cmd ); } \par
This method sets the turning command of the walking controller. A
command $-1$ indicates turning left, $1$ indicates turning right and $0$
indicated no turning.
\item{\tt void setUpside( bool down ); } \par
This method can be used to set the walking controllers assumption
about the robots body orientation. {\tt setUpside( false ) } configures
upside down operation.
\item{\tt void setParams( WalkParam\_t *p ); } \par
This method can be used to set all the walking parameters, modulating
the walking behavior. The parameters are updated only at
half cycles. Moreover, the pointer to the parameter settings is stored,
so any subsequent changes to the contents of {\tt *p} will also change
walking behavior.
\item{\tt void getParams( WalkParam\_t *p ); } \par
This method returns the current walking parameters in {\tt *p }. 
\item{\tt bool isDone( void ); } \par
This method returns a flag indicating whether the walking controller
is done with its operation, which means that both speed and turn
commands are zero and both tripods are lowered.
\end{itemize}

\configsymbols\ None.

\subsection{The \RHexWalker\ Module}
\label{sec:rhexwalker}

\begin{moduleheader}
\classname{\RHexWalker} \mline
\modulebase{\Module} \mline
\modulename{rhexwalker} \mline
\moduleflags{single-user, non-polling}
\usedmodules{\WalkMachine, \SlopeEstimator.}
\end{moduleheader}

The \RHexWalker\ module is a controller which implements variable speed
two-stroke alternating tripod gait with inclination compensation and
variable speed alternating triwheel gait. It accomplishes this by modulating
parameter settings of \WalkMachine, in a way that is adjustable through
configuration files.

The alternating tripod gait is the standard walking algorithm for RHex,
which is used when it has non-circular legs mounted. Section
\ref{sec:walk_machine} summarizes the parameters which modulate the walking
behavior. The RHexWalker module supports variable speed walking, but
choosing parameter values from a piecewise linear interpolation of several
experimentally tuned parameter sets. Several configuration variables define
these various sets and how the interpolation is achieved.

The inclination compensation of the walking gait involves adjusting some of
the walking parameters based on an estimate of the slope through the
\SlopeEstimator\ module. When it is disabled, the leg offsets can also be
manually adjusted through the {\tt setLegOffset()} method call.

The triwheel gait can be used when RHex has circular legs mounted. This gait
also relies on using \WalkMachine, but the parameters are chosen in a very
particular manner. First of all, three of the parameters are kept constant
(with the exception of differential turning): duty factor at $0.5$, leg
offset at $0$ and sweep angle at $\pi$. The period of leg rotation
determines the effective speed, and that parameter is modulated through the
{\tt setForwardCommand()} method. Note that the triwheel gait does not have
inclination compensation and the leg offset cannot be changed. Once again,
the behavior of the speed control and the gait can be adjusted through the
configuration file (see symbol descriptions below).\\

\constructors

\begin{itemize}
\item{\tt RHexWalker( void ); }
\end{itemize}

\localinterface

\begin{itemize}
\item{\tt void setGait( WalkingGait newGait ); }\par
This method sets the walking gait to be used by \RHexWalker. Two gaits are
currently supported {\tt TWO\_STROKE\_TRIPOD\_GAIT} and {\tt
  TRIWHEEL\_GAIT}. The new gait setting becomes effective next time the
module is activated, so changing gaits during walking is not possible.

\item{\tt void setForwardCommand( float cmd ); }\par
This method sets the forward command of the walking controller in the range
$[-1, 1]$. It supports variable speed walking, where the magnitude of the
forward speed command determines the \WalkMachine\ parameters to be used
based on configuration file settings and the current gait (see below).

\item{\tt void setTurnCommand( float cmd ); }\par
This method sets the turning command of the walking controller in the range
$[-1, 1]$. For differential turning, it performs on/off control. For turning
in place, tt supports variable speed turning, where the magnitude of the
speed command determines the \WalkMachine\ parameters to be used
based on configuration file settings and the current gait (see below).

\item{\tt void inclineEnable( bool enable ); }\par
This method can be used to enable/disable the inclination compensation for
the alternating tripod gait. Enabling the inclination compensation overrides
the manual leg offset setting.

\item{\tt void setLegOffset( float offset ); }\par
This method manually sets the leg offsets for all the legs. It is effective
only when the inclination compensation is disabled.

\item{\tt void setUpside( bool down ); }\par
This method can be used to set the walking controllers assumption
about the robots body orientation. {\tt setUpside( false ) } configures
upside down operation.

\item{\tt bool isDone( void ); }\par
This method returns a flag indicating whether the walking controller is
ready to exit or not. Supervisors should check this flag before deactivating
RHexWalker.
\end{itemize}

\configsymbols

\begin{itemize}
\symbolitem{\tt walk\_speeds}\par
This numeric array symbol gives a set of ``speed setting'' indices, to which
the parameter settings of the alternating tripod controller will
correspond. This way, the user can configure experimentally tuned speed
settings to irregularly spaced intervals. For example, if you have
parameters for a very slow walk, and then two other parameter sets for fast
walking with slightly different speeds, {\tt walk\_speeds = \{ 1.0 9.0 10.0
  \}} would be a good setting. Most of the symbols below with the prefix
{\tt walk} must have the same number of array elements as this symbol.

If {\tt walk\_speeds} is not defined, only the first elements of the array
symbols below are used for all speed settings and the variable speed feature
is disabled.

\symbolitem{\tt walk\_tripod\_time}\par
This numeric array symbol configures the time spent to prepare the tripod
before the walking starts as well as to bring the upper tripod down when it
stops. Each of the array entries correspond to one of the speed indices and
for a given speed setting, the value of this parameter is linearly
interpolated between the two enclosing values. If {\tt walk\_speeds} is not
defined, only the first number in the array is used for the parameter.

\symbolitem{\tt walk\_cpg\_period}\par
This numeric array symbol configures the rotation period of all the
legs. The semantics of the array elements are the same as the {\tt
  walk\_tripod\_time} symbol.

\symbolitem{\tt walk\_duty\_factor}\par
This numeric array symbol configures the ratio of the slow period of all the
legs to the overall rotational period. The semantics of the array elements
are the same as the {\tt walk\_tripod\_time} symbol.

\symbolitem{\tt walk\_sweep\_angle}\par
This numeric array symbol configures the angle span of the slow leg rotation
phase. The semantics of the array elements are the same as the {\tt
  walk\_tripod\_time} symbol.

\item{\tt walk\_leg\_offset\_forward, walk\_leg\_offset\_backward}\par
\symbolindex{\tt walk\_leg\_offset\_forward}
\symbolindex{\tt walk\_leg\_offset\_backward}
These numeric array symbols configure the angular offset of the whole leg
profile with respect to the vertical. The semantics of the array elements
are the same as the {\tt walk\_tripod\_time} symbol.

\symbolitem{\tt walk\_cubic}
This numeric symbol defines whether cubic smoothing is enabled for the
profile generation of the alternating tripod walking controller.

\symbolitem{\tt walk\_smooth\_factor}\par
If the cubic smoothing is enabled, this numeric array symbol configures what
percentage of the leg rotation period should be used for the smoothing. Once
again, the corresponding \WalkMachine\ parameter is interpolated as in the
{\tt walk\_tripod\_time} symbol.

\item{\tt walk\_turn\_offset, walk\_turn\_sweep\_angle,
    walk\_turn\_duty\_factor}\par
\symbolindex{\tt walk\_turn\_offset}
\symbolindex{\tt walk\_turn\_sweep\_angle}
\symbolindex{\tt walk\_turn\_duty\_factor}
These numeric array symbols configure how much differential adjustment will
be made to contralateral legs during differential turning for the
corresponding parameters.  The interpolation of the parameters
are done the same as they were for {\tt walk\_tripod\_time} symbol.

\symbolitem{\tt triwheel\_tripod\_time}\par
This numerical symbol tripod preparation and lowering time for the triwheel
gait. This gait does not incorporate parameter interpolation, so this symbol
is only a number.

\item{\tt triwheel\_min\_cpg\_period, triwheel\_max\_cpg\_period}\par
\symbolindex{\tt triwheel\_min\_cpg\_period}
\symbolindex{\tt triwheel\_max\_cpg\_period}
These numerical symbols configure the minimum and the maximum leg rotation
period for the variable speed control of the triwheel gait. Based on the
forward command setting, \RHexWalker\ chooses a value in between to obtain
different walking speeds.

\symbolitem{\tt triwheel\_turn\_sweep\_angle}\par
This numerical symbol configures the differential adjustment in the sweep
angles of contralateral legs to obtain differential turning in the triwheel
controller. Once again, there is no interpolation involved, so this symbol
does not require an array input.
\end{itemize}

\subsection{The \StandMachine\ Module}
\label{sec:stand_machine}

\begin{moduleheader}
\classname{\StandMachine} \mline
\modulebase{\StateMachine} \mline
\modulename{standmachine} \mline
\moduleflags{single-user, non-polling}
\usedmodules{\PositionControl, \EncoderReader.}
\end{moduleheader}

The \StandMachine\ module is a controller which stands the robot up from a
stationary position. Once activated, it commands the legs first to go to a
horizontal position, and then go downwards in synchrony to push the body
up. \\

\constructors

\begin{itemize}
\item{\tt StandMachine( void ); }
\end{itemize}

\localinterface

\begin{itemize}
\item{\tt void setParams( StandParam\_t *p ); } \par
This method can be used to set all the standing parameters. See
\S\ref{sec:special_types} for an explanation of the data type {\tt
  StandParam\_t}.
\item{\tt void getParams( StandParam\_t *p ); } \par
This method returns the current standing parameters in {\tt *p }.
\item{\tt bool isDone( void ); } \par
This method returns a flag indicating whether the standing controller
is done with its operation, which means that all leg targets have
reached the vertical position.
\end{itemize}

\configsymbols

\begin{itemize}
\symbolitem{\tt stand\_one\_time}\par
This numeric symbol configures the time in seconds for the legs to get into
horizontal position.
\symbolitem{\tt stand\_two\_time}\par
This numeric symbol configures the time in seconds for the legs to get lift
the body up from their horizontal position.
\symbolitem{\tt stand\_adj\_time}\par
This numeric symbol configures the period between succesive leg offset
adjustments at the end of standing. Primarily used for runtime adjustment
leg offset for things like inclination compensation. The {\tt
  setLegOffset()} should be used for such compensation actions.
\symbolitem{\tt stand\_kp}\par
This numeric array variable configures the proportional gains for the legs
during standing.
\symbolitem{\tt stand\_kd}\par
This numeric array variable configures the derivative gains for the legs
during standing.
\end{itemize}

\subsection{The \SitMachine\ Module}
\label{sec:sit_machine}

\begin{moduleheader}
\classname{\SitMachine} \mline
\modulebase{\StateMachine} \mline
\modulename{sitmachine} \mline
\moduleflags{single-user, non-polling}
\usedmodules{\PositionControl, \EncoderReader.}
\end{moduleheader}

The \SitMachine\ module is a controller which makes the robot sit down from
a standing position. Once activated, the algorithm commands the legs to go
to a horizontal position, lowering the robot to the ground. \\

\constructors

\begin{itemize}
\item{\tt SitMachine( void ); }
\end{itemize}

\localinterface

\begin{itemize}
\item{\tt bool isDone( void ); } \par
This method returns a flag indicating whether the sitting controller
is done with its operation, which means that all leg targets have
reached the horizontal position.
\end{itemize}

\configsymbols

\begin{itemize}
\symbolitem{\tt sit\_time}\par
Time to bring the legs to their horizontal position (sec).
\end{itemize}

\section{Utility Modules}

\subsection{The \DataLogger\ Abstract Base Class}

\begin{moduleheader}
\classname{\DataLogger} \mline
\modulebase{\Module} \mline
\moduleflags{single-user, non-polling}
\usedmodules{None}
\end{moduleheader}

The DataLogger class is a general purpose utility to store data vectors in
memory while the robot is operational and save them later to files in a
variety of data formats. It is defined as an abstract base class. As a
consequence, it cannot be directly instantiated. Instead, a new class must
be derived from DataLogger, choosing the size of the data vector and the
contents of the log.

The period of the module when added with the {\tt MMAddModule()} function
call determines the frequency at which the data vector is logged. Moreover,
the {\tt logging\_period} symbol provides means of configuring this frequency
independent of the module's own period. Please see the symbol explanation
below for details.

\constructors

\begin{itemize}
\item{\tt DataLogger( void );} \par

This constructor creates a DataLogger with the default name "datalogger",
index 0 and a data vector length of 1. These values, except the record
length, cannot be changed once the class is instantiated, which is also true
for all the other constructors of DataLogger.

\item{\tt DataLogger( int reclen );} \par

This constructor creates a DataLogger with the default name "datalogger",
module index 0 and a data vector length of {\tt reclen}. 

\item{\tt DataLogger( int reclen, int index );} \par

This constructor creates a DataLogger with the default name "datalogger",
the specified {\tt index} and a data vector length of {\tt reclen}.

\item{\tt DataLogger( int reclen, char *name );} \par

This constructor creates a DataLogger with the specified {\tt name},
index 0 and a data vector length of {\tt reclen}.

\item{\tt DataLogger( int reclen, char *name, int index );} \par

This constructor creates a DataLogger with the specified {\tt name},
{\tt index} and a data vector length of {\tt reclen}.

\end{itemize}

\localinterface

\begin{itemize}
\item{\tt virtual void  fillRecord ( float *f ) = 0;} \par
This method is a pure virtual function and needs to be defined by the
inherited classes. It is where the contents of the data vector to be logged
are filled in. The inherited definition must fill in the contents of the
vector {\tt *f} with whichever variables need to be logged.

\item{\tt void reset( void );} \par
This method resets the contents of the DataLogger by deleting all previously
logged data.

\item{\tt void reset( int new\_reclen );} \par
This method also resets the contents of the DataLogger by deleting all
previously logged data. However, in doing so, it also sets the data vector
length of the logger to {\tt new\_reclen}.

\item{\tt void setLoggingPeriod( int p );} \par
This method sets the logging period for the logger, overriding the value
chosen through the {\tt logging\_period} configuration symbol. The unit of
{\tt p} is module manager steps (also milliseconds in the current
implementation).

\item{\tt void exportToFile ( char *basefilename, DL\_OUTPUT\_TYPE type, char** names );} \par
This method dumps the current contents of the logger into a file specified
by {\tt *basefilename}. The suffix {\tt .data} is appended to the end of
this base filename.

The next two arguments are optional and can be left unspecified. The {\tt
type} argument specifies the type of the output file. Currently the only
supported type is {\tt DL\_ASCII}, which outputs columns of ascii
numbers. The default value for this argument (if left unspecified), is {\tt
DL\_ASCII}.

The {\tt names} argument can be used to specify names for the variables that
are being logged. This method then creates a Matlab executable file, {\tt
  basefilename.m}, which, when executed from within Matlab, loads the
corresponding data file and assigns Matlab variables with the specified
names to appropriate columns of the data. If {\tt NULL} is passed, the data
logger does not generate this file. The default value, in case this argument
is left unspecified) is {\tt NULL}.

\item{\tt int exportViaNet( char *basefilename, DL\_OUTPUT\_TYPE type, char
**names );} \par 
This method has very similar functionality to the method {\tt
exportToFile()}. However, after writing the data files to disk, this
function transfers them over the network to the host specified by the {\tt
rhexftphost} config symbol into the directory specified by {\tt
rhexftppath}. Moreover, the configuration symbols {\tt rhexftpuser} and {\tt
rhexftppasswd} must be assigned with the username and password on the target
host.
\end{itemize}

\configsymbols

\begin{itemize}

\symbolitem{\tt logging\_period}\par 

This numeric config symbol specifies the period of logging in milliseconds
(more accurately, module manager steps). It takes into account the module
period of the data logger, but the logging period is not an integer multiple
of the module period, the actual time difference between data samples might
be somewhat uneven. The default is 1.

\symbolitem{\tt rhexftphost}\par
This string config symbol specifies the hostname for FTP transfer of the
data files exported through  {\tt int exportViaNet()}.
\symbolitem{\tt rhexftppath}\par
This string config symbol specifies the target directory for FTP transfer of
the data files exported through  {\tt int exportViaNet()}.
\symbolitem{\tt rhexftpuser}\par
This string config symbol specifies the username on the target host for FTP
transfer of the data files exported through  {\tt int exportViaNet()}.
\symbolitem{\tt rhexftppasswd}\par
This string config symbol specifies the password on the target host for FTP
transfer of the data files exported through  {\tt int exportViaNet()}.
\end{itemize}

\examplecode

Here is an example of how to define a data logger class, inherited from
DataLogger.

\begin{codesegment}
class ExampleLogger : public DataLogger {
  public:
    ExampleLogger( void ) : DataLogger( 2, "examplelogger", 0);

    void fillRecord( float *f ) {
      f[0] = MMReadTime();
      f[1] = sin( 2.0 * M_PI * MMReadTime() );
    }
};
\end{codesegment}

This code segment defines an ExampleLogger module, with name {\tt
"examplelogger"} and index 0, logging the array \{$t$, $\sin t$\}. Note that
this module must be explicitly created and added to the module manager
somewhere (usually in main()) to be of any use.

*WARNING* : If you redefine any of the module functions, you must explicitly
call the corresponding method of the DataLogger base class at the end of the
module function. For example, if ExampleLogger were to redefine the init()
module function, it would have to look like this:

\begin{codesegment}
ExampleLogger::init( void ) {

  ( some code goes here.... )

  DataLogger::init();
};
\end{codesegment}

Otherwise, the data logging functionality will not work properly.

\subsection{The \RHexLogger\ Generic Logging Module}

\begin{moduleheader}
\classname{\RHexLogger} \mline
\modulebase{\Module} \mline
\moduleflags{single-user, non-polling}
\usedmodules{\DataLogger}
\end{moduleheader}

This module provides a variety of logging options, including a set of 140
standard variables, 16 global variables as well as functionality to add new
named variables to be logged at runtime. The set of variables to be logged
each time the program is run is determined by the configuration file (see
the symbol explanations below), so subsequent runs may log different
variables without recompilation. All relevant configuration files should be
read {\em before} the RHexLogger module is created (i.e. before {\tt
RHexAddStdModules()} or similar functions are called).\\

When exporting data to a file (or over the network), RHexLogger also creates
supplies the DataLogger (see above) with the names of the variables,
generating an associated Matlab file to load the variables into the Matlab
workspace with the same names. A very useful feature of RHexLogger is the
ability to include the current date, time and a unique log index as part of
the filename specification in the configuration file using the identifiers
{\em \%d \%t} and {\em \%i}.

An array of 16 floating point global variables {\tt rhexLoggerGlobalVars}
are defined by RHexLogger with the names {\tt user\_defined0} through {\tt
user\_defined15}. When these variables are enabled for logging in the config
file, RHexLogger records the values of these global variables. This facility
can be used for quick debugging, where you can simply assign the value you
would like to be logged to one of these global variables. Unfortunately, it
is not possible to change the names of these variables.

Another custom functionality is the ability to add new variables to be
logged at runtime. This is is mostly for debugging purposes as it requires
recompilation for adding and removing new variables. However it does present
the ability to define variables with custom names, which is useful for
archiving the data files with more meaningful names for custom variables
than {\tt user\_defined5}. This is accomplished through the methods {\tt
  addNewVar()} and {\tt addNewFunc()}. These function must be called {\em
  after} the RHexLogger has been created and added to the module manager
(i.e. after RHexAddStdModules() has been called). The newly added variables
are appended at the end of the variable list configured by {\tt
  rhexloggervars} configuration variable.

One word of caution is that, adding new variables is possible only when the
module is INACTIVE. Moreover, adding new variables resets the contents of
the data logger, erasing all previously logged content because it
effectively changes the size of the data array being logged.

\constructors

\begin{itemize}
\item{\tt RHexLogger( void )}\par
\end{itemize}

\localinterface

\begin{itemize}
\item{\tt void exportToFile( DL\_OUTPUT\_TYPE type = DL\_ASCII ); }\par
Calls the exportToFile method of the underlying \DataLogger\ member,
defaulting to output type DL\_ASCII. The filename is automatically generated
using the contents of the {\tt rhexloggerbasefile} configuration symbol.

\item{\tt void exportViaNet( DL\_OUTPUT\_TYPE type = DL\_ASCII ); }\par
Calls the exportViaNet method of the underlying \DataLogger member,
defaulting to output type DL\_ASCII. NOTE: Since none of the other data 
types have even been instantiated and it is doubtful whether they ever 
will, the user does not even need to worry about it, and can simply
call both export functions with no parameters.

\item{\tt void reset( void ); }\par
Resets the \DataLogger\ and increments the index of the logging
session. During a particular run, \RHexLogger\ keeps track of a unique
logging index to distinguish between separate logging sessions during each
run. Every reset increments this index by one.

\item{\tt void setLoggingPeriod( int period ); }\par
Resets the logging period of the \DataLogger.

\item{\tt Strings getVariables( void ); }\par
Returns a Strings object containing the names of all of the variables
currently being logged by \RHexLogger, in the order of logging.

\item{\tt char* getLastFilename( void ); }\par
Returns a c-string consisting of the filename most recently used by
the logger to export.

\item{\tt void addNewFunc( char* name, RHexLogger::logFnPtr func, int index );}\par

Adds a variable {\tt name} to the logger. Takes a function pointer as its
second parameter. The supplied function should take an integer as its
argument, selecting between different values of the function and return a
{\tt float}. While active, RHexLogger will call this function with the
specified index and record the result.

\item{\tt void addNewVar( char* name, float* var ); }\par
This method adds a new variable named {\tt name} to the logger. The content
of the new variable is specified through the actual address of a {\tt float}
variable, so it is imperative for the user to ensure that this address 
always points to a valid memory location. Although it is possible to pass
pointers to class data members using this interface, be careful not to
deallocate the class or the member.

\end{itemize}

\configsymbols

\begin{itemize}
\symbolitem{\tt rhexloggerbasefile}\par This string symbol is the
basefilename for output files created by \RHexLogger. Two output files,
\emph{basefilename.m} and \emph{basefilename.data} are created. In order to
log multiple sessions without changing this symbol, the following
identifiers can be included anywhere in the string (note: Index is the index
of the logging session):

\begin{center}
\begin{tabular}{l|l|l}
    {\bf Symbol} & {\bf Meaning} & {\bf Format}\\
    \hline
    {\tt \%i} & Index of the logging session (reset count) & {\tt iiii}\\
    {\tt \%d} & Current date & {\tt yymmdd}\\
    {\tt \%t} & Current time & {\tt hhmmss}\\
\end{tabular}
\end{center}

The .m file is a script that imports all of the logged data to MatLab
worspace when executed from within Matlab, assigning them to Matlab
variables with the same names as the logger variables.

\symbolitem{\tt rhexloggervars}\par
This symbol refers to a list of strings, such as \{ ``name1'' ``name2''
... ``lastvarname'' \}. Variable names in the list must be the exact names
that the logger can recognize (see \S\ref{tab:hwlogvars}). If no variables
are specified in the configuration file, then the logger does nothing and
the output file is left blank.
\end{itemize}

NOTE: In the following table, the method calls under the column ``Value to
be logged'' assume that all standard RHexLib modules have been added through
{\tt RHexAddStdModules()}. However, this is not required because RHexLogger
itself uses MMFindModule() to locate all the modules it requires to function
properly. 

%
% This file is part of RHexLib, 
%
% Copyright (c) 2001 The University of Michigan, its Regents,
% Fellows, Employees and Agents. All rights reserved, and distributed as
% free software under the following license.
% 
%  Redistribution and use in source and binary forms, with or without
% modification, are permitted provided that the following conditions are
% met:
% 
% 1) Redistributions of source code must retain the above copyright
% notice, this list of conditions, the following disclaimer and the
% file called "CREDITS" which accompanies this distribution.
% 
% 2) Redistributions in binary form must reproduce the above copyright
% notice, this list of conditions, the following disclaimer and the file
% called "CREDITS" which accompanies this distribution in the
% documentation and/or other materials provided with the distribution.
% 
% 3) Neither the name of the University of Michigan, Ann Arbor or the
% names of its contributors may be used to endorse or promote products
% derived from this software without specific prior written permission.
% 
% THIS SOFTWARE IS PROVIDED BY THE COPYRIGHT HOLDERS AND CONTRIBUTORS
% "AS IS" AND ANY EXPRESS OR IMPLIED WARRANTIES, INCLUDING, BUT NOT
% LIMITED TO, THE IMPLIED WARRANTIES OF MERCHANTABILITY AND FITNESS FOR
% A PARTICULAR PURPOSE ARE DISCLAIMED. IN NO EVENT SHALL THE REGENTS OR
% CONTRIBUTORS BE LIABLE FOR ANY DIRECT, INDIRECT, INCIDENTAL, SPECIAL,
% EXEMPLARY, OR CONSEQUENTIAL DAMAGES (INCLUDING, BUT NOT LIMITED TO,
% PROCUREMENT OF SUBSTITUTE GOODS OR SERVICES; LOSS OF USE, DATA, OR
% PROFITS; OR BUSINESS INTERRUPTION) HOWEVER CAUSED AND ON ANY THEORY OF
% LIABILITY, WHETHER IN CONTRACT, STRICT LIABILITY, OR TORT (INCLUDING
% NEGLIGENCE OR OTHERWISE) ARISING IN ANY WAY OUT OF THE USE OF THIS
% SOFTWARE, EVEN IF ADVISED OF THE POSSIBILITY OF SUCH DAMAGE.

%%%%%%%%%%%%%%%%%%%%%%%%%%%%%%%%%%%%%%%%%%%%%%%%%%%%%%%%%%%%%%%%%%%%%%
% $Id: RHexlogtable.tex,v 1.4 2001/07/20 15:14:06 ulucs Exp $
%
% tables containing information on available variables for logging
%
% Created       : Laura McWilliams, 01/06/2001
% Last Modified : Laura McWilliams, 06/27/2001
%
%%%%%%%%%%%%%%%%%%%%%%%%%%%%%%%%%%%%%%%%%%%%%%%%%%%%%%%%%%%%%%%%%%%%%%

\begin{center}
\begin{longtable}{|l|l|}
  \caption{\label{tab:hwlogvars}Logger Variable Names}
      %\hline
%      {\bf Name in .rc file} & {\bf Value to be logged}\\
      \endfirsthead
      \caption[]{\emph{continued}}\\
      \hline \endhead   
      \hline
      \multicolumn{2}{r}{\emph{continued on next page}}
      \endfoot
      \hline \endlastfoot
      \hline
      {\bf Name in .rc file} & {\bf Value to be logged}\\
      \hline
      \multicolumn{2}{|c|}{\emph{General purpose variables}}\\
      \hline
      time & MMReadTime()\\
      relative\_time & MMReadTime() - timeOfLastReset\\
      \hline 
      \multicolumn{2}{|c|}{\emph{Module-related variables}}\\
      \hline 
      mod\_enc\_speed0 & encreader[0]-$>$getSpeed()\\
      mod\_enc\_speed1 & encreader[1]-$>$getSpeed()\\
      mod\_enc\_speed2 & encreader[2]-$>$getSpeed()\\
      mod\_enc\_speed3 & encreader[3]-$>$getSpeed()\\
      mod\_enc\_speed4 & encreader[4]-$>$getSpeed()\\
      mod\_enc\_speed5 & encreader[5]-$>$getSpeed()\\
      mod\_enc\_pos0 & encreader[0]-$>$getPosition()\\
      mod\_enc\_pos1 & encreader[1]-$>$getPosition()\\
      mod\_enc\_pos2 & encreader[2]-$>$getPosition()\\
      mod\_enc\_pos3 & encreader[3]-$>$getPosition()\\
      mod\_enc\_pos4 & encreader[4]-$>$getPosition()\\
      mod\_enc\_pos5 & encreader[5]-$>$getPosition()\\
      mod\_analogout0 & analogout[0]-$>$getValue()\\
      mod\_analogout1 & analogout[1]-$>$getValue()\\
      mod\_analogout2 & analogout[2]-$>$getValue()\\
      mod\_analogout3 & analogout[3]-$>$getValue()\\
      mod\_analogout4 & analogout[4]-$>$getValue()\\
      mod\_analogout5 & analogout[5]-$>$getValue()\\
      mod\_poscontrol\_pos0 & positioncontrol[0]-$>$getTarget().pos\\
      mod\_poscontrol\_pos1 & positioncontrol[1]-$>$getTarget().pos\\
      mod\_poscontrol\_pos2 & positioncontrol[2]-$>$getTarget().pos\\
      mod\_poscontrol\_pos3 & positioncontrol[3]-$>$getTarget().pos\\
      mod\_poscontrol\_pos4 & positioncontrol[4]-$>$getTarget().pos\\
      mod\_poscontrol\_pos5 & positioncontrol[5]-$>$getTarget().pos\\
      mod\_poscontrol\_vel0 & positioncontrol[0]-$>$getTarget().vel\\
      mod\_poscontrol\_vel1 & positioncontrol[1]-$>$getTarget().vel\\
      mod\_poscontrol\_vel2 & positioncontrol[2]-$>$getTarget().vel\\
      mod\_poscontrol\_vel3 & positioncontrol[3]-$>$getTarget().vel\\
      mod\_poscontrol\_vel4 & positioncontrol[4]-$>$getTarget().vel\\
      mod\_poscontrol\_vel5 & positioncontrol[5]-$>$getTarget().vel\\
      mod\_spdfil0 & speedfilter[0]-$>$getSpeed()\\
      mod\_spdfil1 & speedfilter[1]-$>$getSpeed()\\
      mod\_spdfil2 & speedfilter[2]-$>$getSpeed()\\
      mod\_spdfil3 & speedfilter[3]-$>$getSpeed()\\
      mod\_spdfil4 & speedfilter[4]-$>$getSpeed()\\
      mod\_spdfil5 & speedfilter[5]-$>$getSpeed()\\
      \hline
      \multicolumn{2}{|c|}{\emph{Hardware-related Variables}}\\
      \hline
      hw\_accel\_x & hw-$>$accels-$>$read(AccelHW::AXIS\_X)\\
      hw\_accel\_y & hw-$>$accels-$>$read(AccelHW::AXIS\_Y)\\
      hw\_accel\_z & hw-$>$accels-$>$read(AccelHW::AXIS\_Z)\\
      hw\_gyro\_rate\_x & hw-$>$gyros-$>$read(AccelHW::AXIS\_X)\\
      hw\_gyro\_rate\_y & hw-$>$gyros-$>$read(AccelHW::AXIS\_Y)\\
      hw\_gyro\_rate\_z & hw-$>$gyros-$>$read(AccelHW::AXIS\_Z)\\
      hw\_encoders0 & hw-$>$encoders-$>$read(0)\\
      hw\_encoders1 & hw-$>$encoders-$>$read(1)\\
      hw\_encoders2 & hw-$>$encoders-$>$read(2)\\
      hw\_encoders3 & hw-$>$encoders-$>$read(3)\\
      hw\_encoders4 & hw-$>$encoders-$>$read(4)\\
      hw\_encoders5 & hw-$>$encoders-$>$read(5)\\
      hw\_analogin0 & hw-$>$analogIO-$>$read(0)\\
      hw\_analogin1 & hw-$>$analogIO-$>$read(1)\\
      hw\_analogin2 & hw-$>$analogIO-$>$read(2)\\
      hw\_analogin3 & hw-$>$analogIO-$>$read(3)\\
      hw\_analogin4 & hw-$>$analogIO-$>$read(4)\\
      hw\_analogin5 & hw-$>$analogIO-$>$read(5)\\
      hw\_analogin6 & hw-$>$analogIO-$>$read(6)\\
      hw\_analogin7 & hw-$>$analogIO-$>$read(7)\\
      hw\_analogin8 & hw-$>$analogIO-$>$read(8)\\
      hw\_analogin9 & hw-$>$analogIO-$>$read(9)\\
      hw\_analogin10 & hw-$>$analogIO-$>$read(10)\\
      hw\_analogin11 & hw-$>$analogIO-$>$read(11)\\
      hw\_analogin12 & hw-$>$analogIO-$>$read(12)\\
      hw\_analogin13 & hw-$>$analogIO-$>$read(13)\\
      hw\_analogin14 & hw-$>$analogIO-$>$read(14)\\
      hw\_analogin15 & hw-$>$analogIO-$>$read(15)\\
      hw\_digitalio0 & hw-$>$digitalIO-$>$getByte(0)\\
      hw\_digitalio1 & hw-$>$digitalIO-$>$getByte(1)\\
      hw\_digitalio2 & hw-$>$digitalIO-$>$getByte(2)\\
      hw\_digitalio3 & hw-$>$digitalIO-$>$getByte(3)\\
      hw\_digitalio4 & hw-$>$digitalIO-$>$getByte(4)\\
      hw\_digitalio5 & hw-$>$digitalIO-$>$getByte(5)\\
      hw\_digitalio6 & hw-$>$digitalIO-$>$getByte(6)\\
      hw\_timers0 & hw-$>$timers-$>$read(0)\\
      hw\_timers1 & hw-$>$timers-$>$read(1)\\
      hw\_timers2 & hw-$>$timers-$>$read(2)\\
      hw\_timers3 & hw-$>$timers-$>$read(3)\\
      hw\_timers4 & hw-$>$timers-$>$read(4)\\
      hw\_timers5 & hw-$>$timers-$>$read(5)\\
      hw\_timers6 & hw-$>$timers-$>$read(6)\\
      hw\_timers7 & hw-$>$timers-$>$read(7)\\
      hw\_timers8 & hw-$>$timers-$>$read(8)\\
      hw\_switches0 & hw-$>$switches-$>$read(0)\\
      hw\_switches1 & hw-$>$switches-$>$read(1)\\
      hw\_switches2 & hw-$>$switches-$>$read(2)\\
      hw\_switches3 & hw-$>$switches-$>$read(3)\\
      hw\_switches4 & hw-$>$switches-$>$read(4)\\
      hw\_switches5 & hw-$>$switches-$>$read(5)\\
      hw\_switches6 & hw-$>$switches-$>$read(6)\\
      hw\_switches7 & hw-$>$switches-$>$read(7)\\
      hw\_dials0 & hw-$>$dials-$>$read(0)\\
      hw\_dials1 & hw-$>$dials-$>$read(1)\\
      hw\_dials2 & hw-$>$dials-$>$read(2)\\
      hw\_dials3 & hw-$>$dials-$>$read(3)\\
      hw\_dials4 & hw-$>$dials-$>$read(4)\\
      hw\_dials5 & hw-$>$dials-$>$read(5)\\
      hw\_dcmotor\_volt0 & hw-$>$dcmotors-$>$getVoltage(0)\\
      hw\_dcmotor\_volt1 & hw-$>$dcmotors-$>$getVoltage(1)\\
      hw\_dcmotor\_volt2 & hw-$>$dcmotors-$>$getVoltage(2)\\
      hw\_dcmotor\_volt3 & hw-$>$dcmotors-$>$getVoltage(3)\\
      hw\_dcmotor\_volt4 & hw-$>$dcmotors-$>$getVoltage(4)\\
      hw\_dcmotor\_volt5 & hw-$>$dcmotors-$>$getVoltage(5)\\
      hw\_dcmotor\_curr0 & hw-$>$dcmotors-$>$getCurrent(0)\\
      hw\_dcmotor\_curr1 & hw-$>$dcmotors-$>$getCurrent(1)\\
      hw\_dcmotor\_curr2 & hw-$>$dcmotors-$>$getCurrent(2)\\
      hw\_dcmotor\_curr3 & hw-$>$dcmotors-$>$getCurrent(3)\\
      hw\_dcmotor\_curr4 & hw-$>$dcmotors-$>$getCurrent(4)\\
      hw\_dcmotor\_curr5 & hw-$>$dcmotors-$>$getCurrent(5)\\
      hw\_dcmotor\_backemf0 & hw-$>$dcmotors-$>$getBackEMF(0)\\
      hw\_dcmotor\_backemf1 & hw-$>$dcmotors-$>$getBackEMF(1)\\
      hw\_dcmotor\_backemf2 & hw-$>$dcmotors-$>$getBackEMF(2)\\
      hw\_dcmotor\_backemf3 & hw-$>$dcmotors-$>$getBackEMF(3)\\
      hw\_dcmotor\_backemf4 & hw-$>$dcmotors-$>$getBackEMF(4)\\
      hw\_dcmotor\_backemf5 & hw-$>$dcmotors-$>$getBackEMF(5)\\
      hw\_dcmotor\_temp0 & hw-$>$dcmotors-$>$getTemperature(0)\\
      hw\_dcmotor\_temp1 & hw-$>$dcmotors-$>$getTemperature(1)\\
      hw\_dcmotor\_temp2 & hw-$>$dcmotors-$>$getTemperature(2)\\
      hw\_dcmotor\_temp3 & hw-$>$dcmotors-$>$getTemperature(3)\\
      hw\_dcmotor\_temp4 & hw-$>$dcmotors-$>$getTemperature(4)\\
      hw\_dcmotor\_temp5 & hw-$>$dcmotors-$>$getTemperature(5)\\
      hw\_power\_voltage & hw-$>$power-$>$voltage()\\
      hw\_power\_current & hw-$>$power-$>$current()\\
      \hline
      \multicolumn{2}{|c|}{\emph{User-defined Variables}}\\
      \hline
      user\_defined0 & rhexLoggerGlobalVars[0]\\
      user\_defined1 & rhexLoggerGlobalVars[1]\\ 
      user\_defined2 & rhexLoggerGlobalVars[2]\\
      user\_defined3 & rhexLoggerGlobalVars[3]\\
      user\_defined4 & rhexLoggerGlobalVars[4]\\ 
      user\_defined5 & rhexLoggerGlobalVars[5]\\
      user\_defined6 & rhexLoggerGlobalVars[6]\\
      user\_defined7 & rhexLoggerGlobalVars[7]\\ 
      user\_defined8 & rhexLoggerGlobalVars[8]\\
      user\_defined9 & rhexLoggerGlobalVars[9]\\
      user\_defined10 & rhexLoggerGlobalVars[10]\\ 
      user\_defined11 & rhexLoggerGlobalVars[11]\\
      user\_defined12 & rhexLoggerGlobalVars[12]\\
      user\_defined13 & rhexLoggerGlobalVars[13]\\ 
      user\_defined14 & rhexLoggerGlobalVars[14]\\
      user\_defined15 & rhexLoggerGlobalVars[15]\\
      \hline
\end{longtable}
\end{center}




\subsection{The \VirtualMotor\ Module}
\subsection{The \VirtualInput\ Module}
\subsection{The \PulseWidth\ Module}

%%%%%%%%%%%%%%%%%%%%%%%%%%%%%%%%%%%%%%%%%%%%%%%%
\begin{figure}[!ht]
  \centering
  \rotcaption{Summary of standard modules and state machines in \rhexlib.}
  \begin{sideways}
    \epsfig{figure=ComponentSummary.eps,width=7.5in}
    \label{fig:module_summary}
  \end{sideways}
\end{figure}
%%%%%%%%%%%%%%%%%%%%%%%%%%%%%%%%%%%%%%%%%%%%%%%%
\clearpage


